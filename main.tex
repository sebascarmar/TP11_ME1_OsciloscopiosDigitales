\synctex=1
\input{./Plantilla/Estructura.tex}

\instituto{Universidad Tecnológica Nacional\\[0.2cm]Facultad Regional Córdoba}
\carrera{Ingeniería Electrónica}
\title{Trabajo Práctico de Laboratorio Nº11}
\subtitledoc{Análisis de señales con osciloscopios digitales}
\professor{Ing. Centeno, Carlos \par Ing. Salamero, Martin \par Ing. Guanuco, Luis}
\catedraCaratula{Medidas Electrónicas I}
\catedraHeader{Med. Electrónicas I}
\curso{4R1}
\author{Carreño Marin, Sebastian \par Juarez, Daniel \par Torres, Heber}
\legajo{83497 \par 79111 \par 84640}
\footerauthor{Carreño Marin, Juarez, Torres}
\footerlegajo{83497, 79111, 84640}
%\date{\the\year}
\date{6 de octubre de 2022}



\usepackage{float} %Mejora la interfaz para definir objetos flotantes
\usepackage{caption} %Permite customizar captions en entornos flotantes como figuras y tablas
\usepackage{subcaption} %Provee lo mismo que captions pero para subfiguras y similares

\usepackage{amsmath} %Paquete matemático
\usepackage{amssymb} %Provee flechas, operadores, caracteres especiales, figuras geométricas, etc.
\usepackage{mathtools} %Provee una serie de paquetes que mejoran la apariencia de documentos con matemáticas. Basado en amsmath
\usepackage{siunitx} %Unidades del Sistema Internacional

\usepackage{booktabs} %Mejora la calidad de las tablas
\usepackage{multirow} 
\usepackage{multicol}
\usepackage{array} %Implementación extendida de los entornos array y tabular

\usepackage{enumerate} %Agrega un argumento opcional al entonrno enumerate

\usepackage{soul} %Proporciona espaciado separable con guiones, subrayado, tachado, etc.

%\usepackage{svg} %Permite la integración de gráficos SVG
%\usepackage{blindtext} %Provee comandos para crear textos 'blind' utiles para testear clases y paquetes

\usepackage[spanish]{babel} \addto\captionsspanish{\def\tablename{Tabla}   
\def\listtablename{\'Indice de tablas} } 



\begin{document}
  \maketitle

  \null
  \thispagestyle{empty}
  \pagebreak

  \setcounter{page}{1}
  \tableofcontents
  \newpage
  %\listoffigures
  %\listoftables
  %\pagebreak
  
    \section{Introducción}
    Algunos osciloscopios digitales poseen un módulo matemático, el cual incluye la
    \textit{Transformada Rápida de Fourier} (FFT). Esta herramienta permite analizar
    señales en el dominio de la frecuencia, lo cual es útil en determinadas ocasiones.
    Por otro lado, estos osciloscopios otorgan utilidades en los menú de medidas, que
    permiten caracterizar de forma rápida la señal en cuestión. Todos
    estos detalles son tratados en el presente informe.

    \section{Marco Teórico}

    Realizar análisis de señales en dominio de la frecuencia, posee la ventaja de poder 
    brindar mas información a diferencia de hacerlo bajo el dominio del tiempo. 
    Por ejemplo, se puede visualizar una onda senoidal en el tiempo ,entregada 
    por un generador de funciones, con un osciloscopio y notar una distorsión muy leve 
    en sus picos, casi imperceptible, lo cual, si dicha señal se la analizara bajo el 
    dominio de la frecuencia, quedaria en evidencia clara que la misma no es una señal 
    senoidal pura sino que posee armónicos debido al modo que utilizan los generadodes 
    de funciones para crear dicha señal.
    
    La desventaja de analizar señales en el 
    dominio de la frecuencia es que no se puede obtener información de la fase relativa de 
    la señal bajo análisis, solo es posible obtener la amplitud de las componentes
    espectrales. La otra desventaja a destacar es que, resulta difícil realizar análisis de 
    transitorios rapidos, estos mismos son mas comodos de analizar en el dominio del tiempo.
    
    Para el presente trabajo practico se realizará los ensayos con un osciloscopio digital 
    el cual posee un modulo matemático (\textbf{Math menu}) que cuenta con la transformada 
    rapida de Fourier (FFT). 

    \subsubsection{Analizador de Fourier }
    
    El analizador de Fourier, posee un conjunto de filtros donde cada uno de los 
    mismos se encuentran ligeramente desfasados entre si y repartidos de manera 
    uniforme sobre el margen de frecuencia que se desea analizar, tal como se observa
    en la Figura~\ref{fig:EsqInicialFourier}. El conjunto de filtros puede formar 
    parte de un sistema más complejo, donde sus salidas pueden ser multiplexadas 
    y mostradas en conjunto con un amplificador vertical en una pantalla y a su vez, 
    el multiplexor con un contador, aparte de seleccionar el conjunto de filtros, 
    generar una señal de rampa escalera para posteriormente pasar por un 
    amplificador horizontal y realizar un barrido de eje horizontal de la pantalla 
    para poder visualizar el espectro en frecuencia de una señal. En la 
    Figura~\ref{fig:EsqAnalizadorDeFourierBasico}, se muestra un diagrama en bloques 
    simplificado de un analizador de Fourier el cual, su principal 
    ventaja es que, el análisis se hace prácticamente en forma simultánea en todo 
    el espectro. Pero dichos instrumentos poseen como desventaja una excesiva 
    complejidad del sistema y a su vez poseen baja resolución con un \textit{span} 
    (margen de frecuencia de trabajo) fijo.
    \begin{figure}[H]
        \centering
        \begin{subfigure}[H]{0.48\textwidth}
          \frame{\includegraphics[width=\textwidth]{Imagenes/MarcoTeorico/EsqInicialAnalizDeFourier.png}}
          \caption{Conjunto de filtros.}
          \label{fig:EsqInicialFourier}
        \end{subfigure}
        \hfill 
        \begin{subfigure}[H]{0.45\textwidth}
          \frame{\includegraphics[width=\textwidth]{Imagenes/MarcoTeorico/EsqDeAnalizadorDeFourierBasico.png}}
          \caption{Esquema en bloques simplificador.}
          \label{fig:EsqAnalizadorDeFourierBasico}
        \end{subfigure}
        \caption{Analizador de Fourier básico.}
        \label{fig:AnalizadorDeFourier}
      \end{figure}
       
    
    \subsubsection{Módulo matemático en Osciloscopios Digitales}

    Con la llegada de los osciloscopios digitales con módulo matemático, estos mismos 
    fueron reemplazando los analizadores de Fourier convencionales.
    Dicho módulo posee un algoritmo de Transformada Rápida de Fourier (FFT), donde 
    a partir de las muestras tomadas en el dominio del tiempo de una señal de entrada,
    se realiza la conversion de la señal con la FFT al dominio de la frecuencia.
    El esquema en bloques simplificado se puede observar en la Figura~\ref{fig:MathModuleEnOscil}.
        \begin{figure}[H]
            \centering
            \frame{\includegraphics[width=\textwidth]{Imagenes/MarcoTeorico/MathModuleEnOscilDigital.png}}
            \caption{Módulo matemático en Osciloscopio Digital.}
            \label{fig:MathModuleEnOscil}
        \end{figure}
    
    En el modelo de Osciloscopio utilizado en el presente trabajo práctico (TP) es el  
    (\textit{Tektronix TDS 1001}), cuya presentación de pantalla en modo FFT se observa en 
    la Figura~\ref{fig:MathModoEnTek}.   
        \begin{figure}[H]
            \centering
            \frame{\includegraphics[width=\textwidth]{Imagenes/MarcoTeorico/PresentacionDePantallaFFT.png}}
            \caption{Modo Matemático en Osciloscopio,}
            \label{fig:MathModoEnTek}
        \end{figure}
    Donde se destaca los siguientes puntos 
    \begin{enumerate}
        \item Frecuencia de la linea central de la pantalla.
        \item Escala Vertical en dB/div donde (0 dB = 1 \(V_{RMS}\)).
        \item Escala horizontal en Frec/div.
        \item Velocidad de muestreo. 
        \item Tipo de ventana FFT. 
    \end{enumerate}    

     A continuación, se hace hincapié en los tipos de ventanas que se utilizaran en el 
     presente TP como se observa en la Figura~\ref{fig:VentanasTipos} indicando sus 
     ventajas y desventajas.
        \begin{figure}[H]
            \centering
            \begin{subfigure}[H]{0.45\textwidth}
            \frame{\includegraphics[width=\textwidth]{Imagenes/MarcoTeorico/VentanaRectangular.png}}
            \caption{Ventana Rectangular.}
            \label{fig:VentanaRec}
            \end{subfigure}
            \hfill 
            \begin{subfigure}[H]{0.45\textwidth}
            \frame{\includegraphics[width=\textwidth]{Imagenes/MarcoTeorico/VentanaHamming.png}} 
            \caption{Ventana Hamming.}
            \label{fig:VentanaHamming}
            \end{subfigure}
            \hfill 
            \begin{subfigure}[H]{0.45\textwidth}
            \frame{\includegraphics[width=\textwidth]{Imagenes/MarcoTeorico/VentanaFlattop.png}}
            \caption{Ventana flattop.}
            \label{fig:VentanaFlattop}
            \end{subfigure}

            \caption{Tipos de Ventan para FFT.}
            \label{fig:VentanasTipos}
        \end{figure}
    
    La \textbf{ventana rectangular}, Figura~\ref{fig:VentanaRec} posee como ventaja principal su 
    gran utilidad para medir señales con transitorios rápidos. una desventaja notoria de 
    es que, como la misma es una ventana de apertura y cierre abrupto, esto puede 
    generar la aparición a flancos abruptos que no existen realmente en la señal, causando 
    errores en la lectura de la misma.
    
    La \textbf{ventana Hamming}, Figura~\ref{fig:VentanaHamming} se las considera como una ventana de 
    apertura y cierre suave a diferencia de la rectangular. Esto permite eliminar el problema
    de los flancos abruptos facilitando las mediciones de amplitudes de las componentes 
    espectrales de una señal. Pero como desventaja poseen poca exactitud para realizar 
    mediciones de frecuencias.
    
    Por último la \textbf{ventana flattop}, Figura~\ref{fig:VentanaFlattop} es una solución de compromiso
    entre las dos ventanas previamente mencionadas. 

    \subsubsection*{Uso de Cursores}
        
        Cabe destacar, que también se para realizar las mediciones en los diferentes ensayos
        se utiliza los \textbf{cursores} propios del osciloscopio. Con ellos se puede 
        realizar mediciones en amplitud (dB) o frecuencia tal como se observa en la
        Figura~\ref{fig:CursorTipos}.  
            \begin{figure}[H]
                \centering
                \begin{subfigure}[H]{0.45\textwidth}
                \frame{\includegraphics[width=\textwidth]{Imagenes/MarcoTeorico/CursoresEnMagnitud.png}}
                \caption{Cursores en Magnitud.}
                \label{fig:CursorMag}
                \end{subfigure}
                \hfill 
                \begin{subfigure}[H]{0.45\textwidth}
                \frame{\includegraphics[width=\textwidth]{Imagenes/MarcoTeorico/CursoresEnFrecuencia.png}}
                \caption{Cursores en Frecuencia.}
                \label{fig:CursorFrec}
                \end{subfigure}
                \caption{Tipos de cursores del Osciloscopio Digital.}
                \label{fig:CursorTipos}
            \end{figure}

        Se debe tener presente que cuando e realiza mediciones de magnitudes de las 
        componentes en frecuencia con el osciloscopio, dichas magnitudes dada en 
        decibeles está referenciada a valor de \(V_{RMS}\) y no de amplitud pico de la señal.
                
    

    









    \section{Actividad Práctica}


      \subsection{Análisis de una forma de onda cuadrada}

      \subsection{Análisis de un tren de pulsos}

    Se setea la señal de entrada.

        \begin{figure}[H]
        \centering
        \begin{subfigure}[H]{0.48\textwidth}
          \frame{\includegraphics[width=\textwidth]{Imagenes/ActividadPractica/2AnalisisDeUnTrenDePulsos/Exp2_PeriodoDeLaSeñalDeEntrada.jpeg}}
          \caption{Señal pulsante de entrada, período de $1~ms$.}
        \end{subfigure}
        \hfill
        \begin{subfigure}[H]{0.48\textwidth}
          \frame{\includegraphics[width=\textwidth]{Imagenes/ActividadPractica/2AnalisisDeUnTrenDePulsos/Exp2_AnchoDelPulsoDeEntrada.jpeg}}
          \caption{Ancho del pulso.}
        \end{subfigure}

        \caption{Señal pulsante de entrada.}
        \label{fig:Exp2SeñalPulso}
      \end{figure}

    Se observa en el espectro.

      \begin{figure}[H]
        \centering
          \frame{\includegraphics[width=0.48\textwidth]{Imagenes/ActividadPractica/2AnalisisDeUnTrenDePulsos/Exp2_EspectroDeLaSeñalPulsante.jpeg}}
          \caption{Análisis en frecuencia de la señal de entrada.}
          \label{fig:Exp2SeñalPulsanteEspectro}
      \end{figure}

      Luego se conmuta en los tres tipos de ventana.

      \begin{figure}[H]
        \centering
        \begin{subfigure}[H]{0.48\textwidth}
          \frame{\includegraphics[width=\textwidth]{Imagenes/ActividadPractica/2AnalisisDeUnTrenDePulsos/Exp2_EspectroEnVentanaHannin.jpeg}}
          \caption{Análisis en frecuencia, ventana Hanning.}
        \end{subfigure}
        \hfill
        \begin{subfigure}[H]{0.48\textwidth}
          \frame{\includegraphics[width=\textwidth]{Imagenes/ActividadPractica/2AnalisisDeUnTrenDePulsos/Exp2_EspectroEnVentanaRectangular.png}}
          \caption{Análisis en frecuencia, ventana Rectangular.}
        \end{subfigure}
        \begin{subfigure}[H]{0.48\textwidth}
          \frame{\includegraphics[width=\textwidth]{Imagenes/ActividadPractica/2AnalisisDeUnTrenDePulsos/Exp2_EspectroEnVentanaFlattop.jpeg}}
          \caption{Análisis en frecuencia, ventana Flattop.}
        \end{subfigure}

        \caption{Señal pulsante de entrada.}
        \label{fig:Exp2SeñalPulsanteVentanasEspectro}
      \end{figure}

      Se observan los 5 primeros armónicos en ventana Hanning.

       \begin{figure}[H]
        \centering
        \begin{subfigure}[H]{0.48\textwidth}
          \frame{\includegraphics[width=\textwidth]{Imagenes/ActividadPractica/2AnalisisDeUnTrenDePulsos/Exp2_FrecArmonico1.jpeg}}
          \caption{Frecuencia de la fundamental en ventana Hanning, $f_{1}=950~Hz$.}
        \end{subfigure}
        \hfill
        \begin{subfigure}[H]{0.48\textwidth}
          \frame{\includegraphics[width=\textwidth]{Imagenes/ActividadPractica/2AnalisisDeUnTrenDePulsos/Exp2_FrecArmonico2.jpeg}}
          \caption{Frecuencia de la segunda armónica en ventana Hanning, $f_{2}=1950~Hz$.}
        \end{subfigure}
        \begin{subfigure}[H]{0.48\textwidth}
          \frame{\includegraphics[width=\textwidth]{Imagenes/ActividadPractica/2AnalisisDeUnTrenDePulsos/Exp2_FrecArmonico3.jpeg}}
          \caption{Frecuencia de la tercera armónica en ventana Hanning, $f_{3}=2950~Hz$.}
        \end{subfigure}
        \begin{subfigure}[H]{0.48\textwidth}
          \frame{\includegraphics[width=\textwidth]{Imagenes/ActividadPractica/2AnalisisDeUnTrenDePulsos/Exp2_FrecArmonico4.jpeg}}
          \caption{Frecuencia de la cuarta armónica en ventana Hanning, $f_{4}=3950~Hz$.}
        \end{subfigure}
        \begin{subfigure}[H]{0.48\textwidth}
          \frame{\includegraphics[width=\textwidth]{Imagenes/ActividadPractica/2AnalisisDeUnTrenDePulsos/Exp2_FrecArmonico5.jpeg}}
          \caption{Frecuencia de la quinta armónica en ventana Hanning, $f_{5}=4900~Hz$.}
        \end{subfigure}

        \caption{Medición de frecuencia de armónicos de la señal pulsante en ventana Hanning.}
        \label{fig:Exp2SeñalPulsanteArmonicosEspectro}
      \end{figure}     

      Luego se pasa a Flattop y se miden las frecuencias de los valles.

       \begin{figure}[H]
        \centering
        \begin{subfigure}[H]{0.48\textwidth}
          \frame{\includegraphics[width=\textwidth]{Imagenes/ActividadPractica/2AnalisisDeUnTrenDePulsos/Exp2_FrecValle1Flattop.jpeg}}
          \caption{Frecuencia del primer valle en ventana Flattop, $f_{a}=400~Hz$.}
        \end{subfigure}
        \hfill
        \begin{subfigure}[H]{0.48\textwidth}
          \frame{\includegraphics[width=\textwidth]{Imagenes/ActividadPractica/2AnalisisDeUnTrenDePulsos/Exp2_FrecValle2Flattop.jpeg}}
          \caption{Frecuencia del segundo valle en ventana Rectangular, $f_{b}=1450~Hz$.}
        \end{subfigure}
        \hfill
        \begin{subfigure}[H]{0.48\textwidth}
          \frame{\includegraphics[width=\textwidth]{Imagenes/ActividadPractica/2AnalisisDeUnTrenDePulsos/Exp2_FrecValle3Flattop.jpeg}}
          \caption{Frecuencia del tercer valle en ventana Flattop, $f_{c}=2450~Hz$.}
        \end{subfigure}

        \caption{Medición de frecuencia de valles de la señal pulsante en ventana Flattop.}
        \label{fig:Exp2SeñalPulsanteVallesEspectro}
      \end{figure}     

      Se hace una tabla para la ventana Hanning. Se coloca el cursor 1 en 0Hz y el otro marca el 
      armónico.

      \begin{table}[H]
      \centering
        \begin{tabular}{cccccc} \hline \hline
          \textbf{Cursor 2}               &  $\mathbf{1erArm.}$       & $\mathbf{2daArm.}$        & $\mathbf{3raArm.}$  &   $\mathbf{4taArm.}$ &   $\mathbf{5taArm.}$ \\ \hline
          $\mathbf{\Delta_{fn}~Hz}$       &   $950$                        &    $950$                    &   $950$                & $950$                   & $950$                \\
         \end{tabular}
          \caption{Valores de frecuencia medidos en ventan Hanning.}
          \label{tab:Exp1MedicionesHanning}
      \end{table}

      Se calcula el promedio de las frecuencias.

      \begin{align*}
        \Delta_{fn_{prom}}=\dfrac{\sum{\Delta_{fn}}}{n} \hspace{20pt} \therefore \hspace{20pt} \boxed{\Delta_{fn_{prom}}=950~[Hz]}
      \end{align*}

      Y el período de la onda de pulsos es.

      \begin{align*}
        Periodo~\left( T \right)=\dfrac{1}{\Delta_{fn_{prom}}} \hspace{20pt} \therefore \hspace{20pt} \boxed{Periodo~\left( T \right)=1,053~[ms]}
      \end{align*}

      Se cambia a flattop y se miden los valles.

      \begin{table}[H]
      \centering
        \begin{tabular}{cccc} \hline \hline
          \textbf{Cursor 2}               &  $\mathbf{1erArm.}$       & $\mathbf{2daArm.}$        & $\mathbf{3raArm.}$   \\ \hline
          $\mathbf{\Delta_{fmin}~Hz}$       &   $400$                        &    $1050$                    &   $1000$                     \\
         \end{tabular}
          \caption{Valores de frecuencia medidos en ventan Flattop.}
          \label{tab:Exp1MedicionesFlattop}
      \end{table}  

      También se calcula promedio y período.

      \begin{align*}
        \Delta_{fn_{prom}}=\dfrac{\sum{\Delta_{fn}}}{n} \hspace{20pt} \therefore \hspace{20pt} \boxed{\Delta_{fn_{prom}}=816,67~[Hz]}
      \end{align*}        

      \begin{align*}
        Periodo~\left( T \right)=\dfrac{1}{\Delta_{fn_{prom}}} \hspace{20pt} \therefore \hspace{20pt} \boxed{Periodo~\left( T \right)=1,224~[ms]}
      \end{align*}
      \pagebreak
  \subsection{Obervación de frecuencias producto del aliasing}
    El \textit{aliasing} es un fenómeno que produce componentes de frecuencia falsas, es decir,
    que nos son propias de la señal que se desea medir. Esto, en los osciloscopios digitales,
    ocurre cuando la velocidad de muestreo que se utiliza es insuficiente, es decir, muy baja,
    para la señal que se desea medir.
    
    Por el teorema del muestreo se sabe que la velocidad de muestreo debe ser, como mínimo, el
    doble de la frecuencia máxima de la señal a ser medida.

    \subsubsection{Práctica}
      Con el uso de uno de los generadores se setea una señal \textbf{senoidal} de frecuencia
      $\mathbf{f = 10~kHz}$, y una amplitud acorde a la mitad del valor posible, ya que no es
      algo crítico. Luego, con el menú matemático se eligen las opciones \textbf{FFT}, \textbf{CH1},
      \textbf{Hanning} y \textbf{Zoom X1}. Además, la velocidad de muestreo se setea en 
      $\mathbf{f_s = 25~kSa/s}$. En la Figura~\ref{fig:Exp3Señal10k} se puede observar que la señal 
      es correctamente muestreada, ya que se logra ver su componente de frecuencia.
      
      \begin{figure}[H]
        \centering
        \frame{\includegraphics[width=0.48\textwidth]{Imagenes/ActividadPractica/3ObservacionDeAlias/Exp3_Señal10K.jpeg}}
        \caption{Espectro de señal senoidal de 10~kHz.}
        \label{fig:Exp3Señal10k}
      \end{figure}
      
      Ahora, se procede a cambiar la frecuencia de la señal que se desea medir, y se logra ver que
      a partir de la frecuencia $\mathbf{f = 12,5~kHz}$ empiezan a haber frecuencias falsas. Este
      efecto es esperable, ya que la velocidad de muestreo elegida cumple con el teorema de muestreo
      solo para señales senoidales cuya frecuencia, como máximo, sea la mencionada en este párrafo.

      \begin{figure}[H]
        \centering
        \begin{subfigure}[H]{0.48\textwidth}
          \frame{\includegraphics[width=\textwidth]{Imagenes/ActividadPractica/3ObservacionDeAlias/Exp3_SeñalConAliasing.jpeg}}
          \caption{$f=12,5~kHz$ y $fs=25~kSa/s$.}
          \label{fig:Exp3AliasSeñal12_5kHz}
        \end{subfigure}
        \hfill 
        \begin{subfigure}[H]{0.48\textwidth}
          \frame{\includegraphics[width=\textwidth]{Imagenes/ActividadPractica/3ObservacionDeAlias/Exp3_SeñalConAliasing.jpeg}}
          \caption{$f=15~kHz$ y $fs=25~kSa/s$.}
          \label{fig:Exp3AliasSeñal15kHz}
        \end{subfigure}

        \caption{Espectro de señales con aliasing.}
        \label{fig:Exp3SeñalConAlias}
      \end{figure}

      \subsubsection*{Aliasing con señal cuadrada}
      Se procede a hacer la misma experiencia anterior, pero con una señal cuadrada de $\mathbf{f=10kHz}$.
      Su forma en el tiempo y su espectro se puede ver en la Figura~\ref{fig:Exp3SeñalCuad10k}.

      \begin{figure}[H]
        \centering
        \begin{subfigure}[H]{0.48\textwidth}
          \frame{\includegraphics[width=\textwidth]{Imagenes/logo-utn.png}}
          \caption{$f=10~kHz$ .}
          \label{fig:Exp3SeñalCuadEnTiempo}
        \end{subfigure}
        \hfill 
        \begin{subfigure}[H]{0.48\textwidth}
          \frame{\includegraphics[width=\textwidth]{Imagenes/ActividadPractica/3ObservacionDeAlias/Exp3_SeñalCuadradaSINAlias.jpeg}}
          \caption{$f=15~kHz$ y $fs=25~kSa/s$.}
          \label{fig:Exp3EspectroSeñalCuad}
        \end{subfigure}

        \caption{Espectro de señales con aliasing.}
        \label{fig:Exp3SeñalCuad10k}
      \end{figure}

      Luego, se procede a aumentar la frecuencia de la señal del generador, y también se logra ver los efectos
      del aliasing. En la Figura~\ref{fig:Exp3SeñalCuadConAlias} se muestra lo mencionado.

      \begin{figure}[H]
        \centering
        \begin{subfigure}[H]{0.48\textwidth}
          \frame{\includegraphics[width=\textwidth]{Imagenes/ActividadPractica/3ObservacionDeAlias/Exp3_SeñalConAliasing.jpeg}}
          \caption{$f=12,5~kHz$ y $fs=25~kSa/s$.}
          \label{fig:Exp3AliasSeñalCuad1}
        \end{subfigure}
        \hfill 
        \begin{subfigure}[H]{0.48\textwidth}
          \frame{\includegraphics[width=\textwidth]{Imagenes/ActividadPractica/3ObservacionDeAlias/Exp3_SeñalConAliasing.jpeg}}
          \caption{$f=15~kHz$ y $fs=25~kSa/s$.}
          \label{fig:Exp3AliasSeñalCuad2}
        \end{subfigure}

        \caption{Espectro de señales con aliasing.}
        \label{fig:Exp3SeñalCuadConAlias}
      \end{figure}


      \subsection{Análisis de una señal modulada en amplitud}

      \subsection{Observación de los productos de IMD de tercer orden}

      \subsection{Análisis de una señal modulada en frecuencia}

      \subsection{Análisis de la distorsión armónica producida por un amplificador}
    Se propone analizar un amplificador transistorizado de 4 etapas. Se ha tratado 
    en las experiencias anteriores la \textbf{alinealidad} de éste tipo de dispositivos, 
    para el presente experimento, nuevamente se debe tener en cuenta. 

    Un amplificador tiene comportamiento lineal bajo determinadas condiciones, entre ellas, 
    para pequeña señal y trabajando a lazo cerrado. Sucede que al trabajar en lazo abierto y 
    a máxima excursión, se ingresa en zonas no lineales de la función de transferencia, lo que 
    modifica el comportamiento del amplificador, provocando lo que se denomina 
    \textbf{Distorsión Armónica}.

    En la experiencia se determina el porcentaje de contenido armónico, tanto para lazo 
    abierto como para lazo cerrado, utilizando la 
    herramienta de análisis en frecuencia del osciloscopio. El circuito a implementar 
    se enseña en la Figura~\ref{fig:Exp7EsquemaCircuito}.

      \begin{figure}[H]
        \centering
          \frame{\includegraphics[width=0.8\textwidth]{Imagenes/ActividadPractica/7AnalisisDeDistorisonArmonicaEnAmpli/EsquemaConexionAmplificador.png}}
          \caption{Esquema del circuito a implementar.}
          \label{fig:Exp7EsquemaCircuito}
      \end{figure}

    En la Figura~\ref{fig:Exp7Circuito} se muestran los dispositivos y el 
    instrumental a usar.

      \begin{figure}[H]
        \centering
        \begin{subfigure}[H]{0.48\textwidth}
          \frame{\includegraphics[width=0.8\textwidth]{Imagenes/ActividadPractica/7AnalisisDeDistorisonArmonicaEnAmpli/Amplificador.jpeg}}
          \caption{Amplificador a utilizar.}
          \label{fig:Exp7Amplificador}
        \end{subfigure}
        \hfill 
        \begin{subfigure}[H]{0.48\textwidth}
          \frame{\includegraphics[width=0.8\textwidth]{Imagenes/ActividadPractica/7AnalisisDeDistorisonArmonicaEnAmpli/Equipo.jpeg}}
          \caption{Instrumental a utilizar.}
          \label{fig:Exp7Instrumental}
        \end{subfigure} 

          \caption{Amplificador a utilizar.}
          \label{fig:Exp7Circuito}
      \end{figure} 

    Inicialmente se configura el generador para \textbf{MES} (máxima excursión simétrica), y 
    una frecuencia de $\mathbf{1~kHz}$. Para ello se procede a variar la amplitud del generador hasta 
    el punto donde hay un recorte visible en la señal. La Figura~\ref{fig:Exp7MESLazoAbierto} 
    muestra dicho punto de máxima excursión.

      \begin{figure}[H]
        \centering
          \frame{\includegraphics[width=0.48\textwidth]{Imagenes/ActividadPractica/7AnalisisDeDistorisonArmonicaEnAmpli/Exp7_MESlazoAbierto.png}}
          \caption{Máxima excursión simétrica a lazo abierto.}
          \label{fig:Exp7MESLazoAbierto}
      \end{figure}

      Luego, se procede a hacer un análisis de frecuencia, para ello se configura el osciloscopio nuevamente 
      en modo \textbf{FFT}, \textbf{Flattop}, \textbf{Zoom X10}, y $\mathbf{100~kS/s}$. En éste punto, se procede 
      a realizar la medición de la frecuencia de la fundamental y sus dos primeros armónicos, lo cual se observa en la 
      Figura~\ref{fig:Exp7FrecFundYArmonicas}.

      \begin{figure}[H]
        \centering
        \begin{subfigure}[H]{0.48\textwidth}
          \frame{\includegraphics[width=\textwidth]{Imagenes/ActividadPractica/7AnalisisDeDistorisonArmonicaEnAmpli/Exp7_EspectroLazoAbiertoConCursorEn1k.png}}
          \caption{Frecuencia de la fundamental $f_{fund}=1~kHz$.}
          \label{fig:Exp7FrecFundamental}
        \end{subfigure}
        \hfill 
        \begin{subfigure}[H]{0.48\textwidth}
          \frame{\includegraphics[width=\textwidth]{Imagenes/ActividadPractica/7AnalisisDeDistorisonArmonicaEnAmpli/Exp7_FrecuenciasDeLaSegundaYTercerArmonica.png}}
          \caption{Frecuencias de la segunda y tercer armónica $f_{2}=2~kHz$ y $f_{3}=3~kHz$.}
          \label{fig:Exp7Frec2da3raArmo}
        \end{subfigure}     
        \caption{Análisis espectral del amplificador con señal de $1~kHz$ a lazo abierto.}
        \label{fig:Exp7FrecFundYArmonicas}
      \end{figure}

      En la Figura~\ref{fig:Exp7AmplFundYArmonicasLA}, se miden las amplitudes de la fundamental y sus 
      dos primeros armónicos.

    \begin{figure}[H]
        \centering
        \begin{subfigure}[H]{0.48\textwidth}
          \frame{\includegraphics[width=\textwidth]{Imagenes/ActividadPractica/7AnalisisDeDistorisonArmonicaEnAmpli/Exp7_AmplitudDeLaFundamental.png}}
          \caption{Fundamental $V_{fund}=5,03~dBv$.}
          \label{fig:Exp7AmpFundamentalLA}
        \end{subfigure}
        \hfill 
        \begin{subfigure}[H]{0.48\textwidth}
          \frame{\includegraphics[width=\textwidth]{Imagenes/ActividadPractica/7AnalisisDeDistorisonArmonicaEnAmpli/Exp7_AmplitudDelSegundoArmónico.png}}
          \caption{2da armónica $V_{2da}=-37,4~dBv$.}
          \label{fig:Exp7AmpSegundaLA}
        \end{subfigure}     
        \begin{subfigure}[H]{0.48\textwidth}
          \frame{\includegraphics[width=\textwidth]{Imagenes/ActividadPractica/7AnalisisDeDistorisonArmonicaEnAmpli/Exp7_AmplitudDelTercerArmónicoALazoAbierto.png}}
          \caption{3ra armónica $V_{2da}=-47~dBv$.}
          \label{fig:Exp7AmpTercerLA}
        \end{subfigure}   
        \caption{Amplitud de la fundamental y sus dos primeras armónicas a lazo abierto.}
        \label{fig:Exp7AmplFundYArmonicasLA}
      \end{figure}

      Ahora, se procede a hacer el mismo análisis a lazo cerrado. Para empezar, se restituye el nivel de la 
      fundamental, y se ubican las frecuencias 
      de la misma y sus dos primeros armónicos, ésto se enseña en la Figura~\ref{fig:Exp7FrecLazoCerrado}.

      \begin{figure}[H]
        \centering
          \frame{\includegraphics[width=0.48\textwidth]{Imagenes/ActividadPractica/7AnalisisDeDistorisonArmonicaEnAmpli/Exp7_LazoCerradoFrecuenciasDeSegundoYTercerArmonico.png}}
          \caption{Frecuencia de armónicos a lazo cerrado.}
          \label{fig:Exp7FrecLazoCerrado}
      \end{figure}

      Por consiguiente, se miden las amplitudes de los ya mencionados, como muestra la 
      Figura~\ref{fig:Exp7AmplFundYArmonicasLC}. Para ello se coloca un cursor en el punto 
      a medir, y éste valor es correspondiente a la amplitud en dB referenciada a $1~[V_{rms}]$.

      \begin{figure}[H]
        \centering
        \begin{subfigure}[H]{0.48\textwidth}
          \frame{\includegraphics[width=\textwidth]{Imagenes/ActividadPractica/7AnalisisDeDistorisonArmonicaEnAmpli/Exp7_LazoCerradoAmplitudDelPrimerArmonico.png}}
          \caption{Fundamental $V_{fund}=66~dBv$.}
          \label{fig:Exp7AmpFundamentalLC}
        \end{subfigure}
        \hfill 
        \begin{subfigure}[H]{0.48\textwidth}
          \frame{\includegraphics[width=\textwidth]{Imagenes/ActividadPractica/7AnalisisDeDistorisonArmonicaEnAmpli/Exp7_LazoCerradoAmplitudDelSegundoArmonico.png}}
          \caption{2da armónica $V_{2da}=18,4~dBv$.}
          \label{fig:Exp7AmpSegundaLC}
        \end{subfigure}     
        \begin{subfigure}[H]{0.48\textwidth}
          \frame{\includegraphics[width=\textwidth]{Imagenes/ActividadPractica/7AnalisisDeDistorisonArmonicaEnAmpli/Exp7_LazoCerradoAmplitudDelTercerArmónico.png}}
          \caption{3ra armónica $V_{2da}=16,4~dBv$.}
          \label{fig:Exp7AmpTercerLC}
        \end{subfigure}   
        \caption{Amplitud de la fundamental y sus dos primeras armónicas a lazo cerrado.}
        \label{fig:Exp7AmplFundYArmonicasLC}
      \end{figure}    

      Finalmente se confecciona una tabla de valores con los resultados obtenidos para lazo abierto y 
      lazo cerrado, y se calcula la 
      \textbf{Distorsión armónica} (DA) en porcentaje, que se define como 
        \begin{equation}
          DA_{\%}= \dfrac{V_{arm}}{V_{1}} \cdot 100~, \vspace{10pt} con~V_{arm}=\sqrt[]{V_{2da}^2+V_{3ra}^2}~.
          \label{eqn:DistorsionArmonica}
        \end{equation}
      
      Lo anteriormente dicho se muestra en la 
      Tabla~\ref{tab:Exp7DistArmLA} y la Tabla~\ref{tab:Exp7DistArmLC}.

      \begin{table}[H]
      \centering
        \begin{tabular}{ccccc} \hline \hline
          \textbf{Magnitud}            &   $\mathbf{V_{salida(PAP)}}$  &  $\mathbf{1erArmonica}$  & $\mathbf{2daArmonica}$  & $\mathbf{3raArmonica}$\\ \hline \hline
          \textbf{Frecuencia~[kHz]}    &   $1$                         &    $1$                   &   $2$                   & $3$ \\ \hline
          \textbf{dBv}                 &   $-$                         &    $5,03$                &   $-37,4$                & $-47$ \\ \hline
          \textbf{Tensión~[V]}         &   $4,88$                      &    $1,784$             &   $0,0135$              & $0,00447$\\ \hline \hline
          \end{tabular}
          \caption{Valores obtenidos a Lazo Abierto.}
          \label{tab:Exp7DistArmLA}
      \end{table}

     \begin{table}[H]
      \centering
        \begin{tabular}{ccccc} \hline \hline
          \textbf{Magnitud}            &   $\mathbf{V_{salida(PAP)}}$  &  $\mathbf{1erArmonica}$  & $\mathbf{2daArmonica}$  & $\mathbf{3raArmonica}$\\ \hline \hline
          \textbf{Frecuencia~[kHz]}    &   $1$                         &    $1$                   &   $2$                   & $3$ \\ \hline
          \textbf{dBv}                 &   $-$                         &    $5,03$                &   $-42,6$                & $-44,6$ \\\hline
          \textbf{Tensión~[V]}         &   $4,88$                      &    $1,784$             &   $0,00741$              & $0,00588$\\ \hline \hline
          \end{tabular}
          \caption{Valores obtenidos a Lazo Cerrado.}
          \label{tab:Exp7DistArmLC}
      \end{table}

      Finalmente, se determina la distorsión armónica total a lazo abierto y lazo 
      cerrado. A lazo abierto se tiene que 
      \begin{align*}
        V_{arm_{Lazo Abierto}}=\sqrt[]{(13,5~mV)^2+(4,47~mV)^2} \hspace{20pt} \Longrightarrow \hspace{20pt} V_{arm}=14,22~[mV]~,
      \end{align*}
      y haciendo uso de la ecuación~\ref{eqn:DistorsionArmonica} dicho valor será
      \begin{align*}
        Distorsion~Armonica_{arm_{Lazo Abierto}}= \dfrac{14,22~mV}{1,784~V} \cdot 100 \hspace{20pt} \therefore \hspace{20pt} \boxed{Distorsion_{LA}=0,8\%}~.
      \end{align*}

      Luego se repiten los cálculos para lazo cerrado, obteniéndose 
      \begin{align*}
        V_{arm_{Lazo Cerrado}}=\sqrt[]{(7,41~mV)^2+(5,88~mV)^2} \hspace{20pt} \Longrightarrow \hspace{20pt} V_{arm}=9,46~[mV]~,
      \end{align*}
      y con éste valor se determina la distorsión armónica total, que será 
      \begin{align*}
        Distorsion~Armonica_{arm_{Lazo Cerrado}}= \dfrac{9,46~mV}{1,784~V} \cdot 100 \hspace{20pt} \therefore \hspace{20pt} \boxed{Distorsion_{LC}=0,53\%}~.
      \end{align*}


    \pagebreak
  \section{Conclusiones}


  Respecto a los distintos tipos de ventanas que se encuentran disponibles para la FFT, se puede realizar
  las siguientes afirmaciones:

  \begin{itemize}
    \item \textbf{Hanning:} es útil para las formas de onda periódicas. Permite medir mejor la frecuencia, pero peor
      la amplitud que la ventana Flattop.
    \item \textbf{Flattop:} es útil para formas de onda periódica. Permite medir mejor la amplitud, pero peor
      la frecuencia que la ventana Hanning.
    \item \textbf{Rectangular:} es útil para pulsos o señales transitorias. Es específica para formas de onda
      que no presentan discontinuidades.
  \end{itemize}

  En el experimento 3 se forzó la aparición del aliasing. Esto se dio debido a que la velocidad de muestreo ($25~kSa/s$)
  no cumplía con el teorema del Muestreo a partir de determinada frecuencia de la señal a muestrear (a partir de $12~kHz$).
  Esto significa, por un lado, que la señal no puede ser muestreada de forma correcta, y además, se generan componentes
  de frecuencia falsas que no son propias de la señal original.

  Para concluir, se nombran algunas características sobre los distintos modos de adquisición que posee el osciloscopio
  digital:

  \begin{itemize}
    \item \textbf{Normal}: el osciloscopio muestrea la señal en intervalos de tiempo equidistantes para construir la señal. Este
      modo representa de forma acertada señales analógicas sin transitorios rápidos.
    \item \textbf{Detección de picos}: el osciloscopio detecta los valores máximo y mínimo de la señal de entrada en un intervalo
      de muestreo, y los utiliza para construir la señal. Con este modo se pueden detectar pulsos transitorios, que con el
      modo Normal podrían perderse.
    \item \textbf{Promedio}: el osciloscopio adquiere determinada cantidad de veces la señal, promedia esos datos y construye la
      señal con el resultado. Es útil para reducir el ruido.
  \end{itemize}


  
\end{document}
