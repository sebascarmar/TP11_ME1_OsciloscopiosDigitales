  \pagebreak
  \subsection{Observación de frecuencias producto del aliasing}
    El \textit{aliasing} es un fenómeno que produce componentes de frecuencia falsas, es decir,
    que nos son propias de la señal que se desea medir. Esto, en los osciloscopios digitales,
    ocurre cuando la velocidad de muestreo que se utiliza es insuficiente, es decir, muy baja,
    para la señal que se desea medir.
    
    Por el teorema del muestreo se sabe que la velocidad de muestreo debe ser, como mínimo, el
    doble de la frecuencia máxima de la señal a ser medida.

    \subsubsection{Aliasing con señal senoidal}
      Con el uso de uno de los generadores se setea una señal \textbf{senoidal} de frecuencia
      $\mathbf{f = 10~kHz}$, y una amplitud acorde a la mitad del valor posible, ya que no es
      algo crítico. Luego, con el menú matemático se eligen las opciones \textbf{FFT}, \textbf{CH1},
      \textbf{Hanning} y \textbf{Zoom x1}. Además, la velocidad de muestreo se setea en 
      $\mathbf{f_s = 25~kSa/s}$. En la Figura~\ref{fig:Exp3Señal10k} se puede observar que la señal 
      es correctamente muestreada, ya que se logra ver su componente de frecuencia.
      
      \begin{figure}[H]
        \centering
        \begin{subfigure}[H]{0.40\textwidth}
          \frame{\includegraphics[width=\textwidth]{Imagenes/ActividadPractica/3ObservacionDeAlias/Exp3_Señal10k_25kSa_tiempo.jpeg}}
          \caption{En tiempo.}
        \end{subfigure}
        \hfill
        \begin{subfigure}[H]{0.41\textwidth}
          \frame{\includegraphics[width=\textwidth]{Imagenes/ActividadPractica/3ObservacionDeAlias/Exp3_Señal10k_25kSa_frec.jpeg}}
          \caption{En frecuencia.}
        \end{subfigure}

        \caption{Muestreo de señal de $10~kHz$ con $f_s=25~kSa/s$ (sin aliasing).}
        \label{fig:Exp3Señal10k}
      \end{figure}
      
      Ahora, se procede a cambiar la frecuencia de la señal que se desea medir, y se logra ver que
      a partir de la frecuencia $\mathbf{f = 12,5~kHz}$ empiezan a haber frecuencias falsas. Este
      efecto es esperable, ya que la velocidad de muestreo elegida cumple con el teorema de muestreo
      solo para señales senoidales cuya frecuencia, como máximo, sea la mencionada en este párrafo. En
      la Figura~\ref{fig:Exp3SeñalConAlias} se logra ver lo expresado para una señal de $\mathbf{f=14,3~kHz}$.



      \begin{figure}[H]
        \centering
        \begin{subfigure}[H]{0.42\textwidth}
          \frame{\includegraphics[width=\textwidth]{Imagenes/ActividadPractica/3ObservacionDeAlias/Exp3_Señal14k_25kSa_tiempo.jpeg}}
          \caption{En tiempo.}
        \end{subfigure}
        \hfill 
        \begin{subfigure}[H]{0.40\textwidth}
          \frame{\includegraphics[width=\textwidth]{Imagenes/ActividadPractica/3ObservacionDeAlias/Exp3_Señal14k_25kSa_frec.jpeg}}
          \caption{En frecuencia.}
        \end{subfigure}

        \caption{Muestreo de señal senoidal de $14~kHz$ con $f_s=25~kSa/s$ (con aliasing).}
        \label{fig:Exp3SeñalConAlias}
      \end{figure}

      \subsubsection{Aliasing con señal cuadrada}
      Se procede a hacer la misma experiencia anterior, pero con una señal cuadrada de $\mathbf{f=10kHz}$.
      Su forma en el tiempo y su espectro se puede ver en la Figura~\ref{fig:Exp3SeñalCuad10k}.

      \begin{figure}[H]
        \centering
        \begin{subfigure}[H]{0.41\textwidth}
          \frame{\includegraphics[width=\textwidth]{Imagenes/ActividadPractica/3ObservacionDeAlias/Exp3_SeñalCuadradaSINAlias_tiempo.jpeg}}
          \caption{$f=10~kHz$ .}
          \label{En tiempo.}
        \end{subfigure}
        \hfill 
        \begin{subfigure}[H]{0.38\textwidth}
          \frame{\includegraphics[width=\textwidth]{Imagenes/ActividadPractica/3ObservacionDeAlias/Exp3_SeñalCuadradaSINAlias.jpeg}}
          \caption{En frecuencia.}
          \label{fig:Exp3EspectroSeñalCuad}
        \end{subfigure}

        \caption{Muestreo de una señal cuadrada de $10~kHz$ con $f_s=25~kSa/s$ (sin aliasing).}
        \label{fig:Exp3SeñalCuad10k}
      \end{figure}

      Luego, se procede a aumentar la frecuencia de la señal del generador, y también se logra ver los efectos
      del aliasing. En la Figura~\ref{fig:Exp3SeñalCuadConAlias} se muestra lo mencionado, en donde aparecen
      frecuencias que no son propias de la señal cuadrada.

      \begin{figure}[H]
        \centering
        \begin{subfigure}[H]{0.41\textwidth}
          \frame{\includegraphics[width=\textwidth]{Imagenes/ActividadPractica/3ObservacionDeAlias/Exp3_SeñalCuadradaCONAlias_tiempo.jpeg}}
          \caption{En tiempo.}
          \label{fig:Exp3AliasSeñalCuad1}
        \end{subfigure}
        \hfill 
        \begin{subfigure}[H]{0.39\textwidth}
          \frame{\includegraphics[width=\textwidth]{Imagenes/ActividadPractica/3ObservacionDeAlias/Exp3_SeñalCuadradaCONAlias.jpeg}}
          \caption{En frecuencia.}
          \label{fig:Exp3AliasSeñalCuad2}
        \end{subfigure}

        \caption{Espectro de señal cuadrada de $13~kHz$ con $f_s=25~kSa/s$ (con aliasing).}
        \label{fig:Exp3SeñalCuadConAlias}
      \end{figure}

