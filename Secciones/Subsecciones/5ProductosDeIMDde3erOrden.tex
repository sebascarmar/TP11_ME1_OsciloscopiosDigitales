  \subsection{Observación de los productos de IMD de tercer orden}
    En la experiencia anterior se hace uso de un diodo para poder generar la modulación de
    amplitud. El circuito en cuestión se puede ver en la Figura~\ref{fig:ModuladorAM} de
    la Sección~\ref{sec:Exp4_AM}.

    Este dispositivo tiene un comportamiento alineal, el cual se puede modelar
    de la siguiente  forma

    \vspace{-10pt}
    \begin{equation*}
      v_{AM} = k_a(v_{G1}+v_{G2}) + k_b(v_{G1}+v_{G2})^2 + k_c(v_{G1}+v_{G2})^3 + \cdots~,
    \end{equation*}
    donde es de especial interés la región cuadrática de este dispositivo, ya que,
    debido a esta, se obtiene la modulación de amplitud buscada. Si se 
    desarrolla el término correspondiente (el cuadrático), se logra justificar lo
    mencionado
    
    \vspace{-10pt}
    \begin{equation*}
      v_{AM} = \cdots + k_b(v_{G1}^2 + 2\cdot{v_{G1}v_{G2}} + v_{G2}^2) + \cdots ~.
    \end{equation*}

    Por el contrario, las \textbf{alinealidades de orden superior} dan como resultado \textbf{productos
    de intermodulación} (\textbf{IMD}), los cuales son efectos no deseados. La alinealidad más
    importante suele ser la IMD de tercer orden, por lo cual, ahora se procede a determinar el
    \textbf{rechazo de IMD de tercer orden} del circuito modulador utilizado.

    Para ello, se setean ambos generadores \textbf{G1} y \textbf{G2} a una frecuencia de 
    \textbf{f=50~kHz}, y a \textbf{igual amplitud}. Luego, se observan las señales de salida del circuito 
    dejando un solo generador encendido a la vez. Los resultados se encuentran en las 
    Figuras~\ref{fig:SalidaCircuitConG1Encendido} y \ref{fig:SalidaCircuitConG2Encendido}.


    \begin{figure}[H]
      \centering
      \begin{subfigure}[H]{0.44\textwidth}
        \frame{\includegraphics[width=\textwidth]{Imagenes/ActividadPractica/5ProductosDeIMPde3erOrden/Exp5_SeñalSalidaConG1Autoset_Tiempo.jpeg}}
        \caption{En tiempo.}
      \end{subfigure}
      \hfill 
      \begin{subfigure}[H]{0.44\textwidth}
        \frame{\includegraphics[width=\textwidth]{Imagenes/ActividadPractica/5ProductosDeIMPde3erOrden/Exp5_SeñalSalidaConG1_Frecuencia.jpeg}}
        \caption{En frecuencia.}
      \end{subfigure}

      \caption{Salida del circuito con el generador G1 encendido.}
      \label{fig:SalidaCircuitConG1Encendido}
    \end{figure}


    \begin{figure}[H]
      \centering
      \begin{subfigure}[H]{0.44\textwidth}
        \frame{\includegraphics[width=\textwidth]{Imagenes/ActividadPractica/5ProductosDeIMPde3erOrden/Exp5_SeñalSalidaConG2Autoset_Tiempo.jpeg}}
        \caption{En tiempo.}
      \end{subfigure}
      \hfill 
      \begin{subfigure}[H]{0.43\textwidth}
        \frame{\includegraphics[width=\textwidth]{Imagenes/ActividadPractica/5ProductosDeIMPde3erOrden/Exp5_SeñalSalidaConG2_Frecuencia.jpeg}}
        \caption{En frecuencia.}
      \end{subfigure}

      \caption{Salida del circuito con el generador G2 encendido.}
      \label{fig:SalidaCircuitConG2Encendido}
    \end{figure}

    A continuación, se encienden ambos generadores y se ajusta el tiempo de muestreo y se habilita el \textbf{Zoom x10}
    para obtener una mejor visualización. Luego, se procede a separar ambas señales un ancho de $\mathbf{\triangle f=  2~kHz}$,
    quedando una de ellas en $\mathbf{f_1=49~kHz}$ y la otra en $\mathbf{f_2=51~kHz}$. En la
    Figura~\ref{fig:SalidaConAmbosGeneradores} se puede ver lo explicado en este párrafo.

    \begin{figure}[H]
      \centering
      \begin{subfigure}[H]{0.48\textwidth}
        \frame{\includegraphics[width=\textwidth]{Imagenes/ActividadPractica/5ProductosDeIMPde3erOrden/Exp5_SeñalSalidaConLosDosGeneradores_Frec.jpeg}}
        \caption{Ambas a la misma frecuencia.}
      \end{subfigure}
      \hfill 
      \begin{subfigure}[H]{0.48\textwidth}
        \frame{\includegraphics[width=\textwidth]{Imagenes/ActividadPractica/5ProductosDeIMPde3erOrden/Exp5_SeparacionDeSeñalesDe2KHz.jpeg}}
        \caption{Con separación de $2~kHz$.}
      \end{subfigure}

      \caption{Espectro de la señal de salida con ambas señales inyectadas al circuito.}
      \label{fig:SalidaConAmbosGeneradores}
    \end{figure}

    Las componentes de $\mathbf{47~kHz}\  (2f_1 - f2)$ y $\mathbf{53~kHz}\  (2f_2 - f1)$ son los productos de IMD de 
    tercer orden. Para obtener el rechazo a los mismos, se realiza la diferencia en amplitud entre estas y 
    \textbf{f\textsubscript{1}} y \textbf{f\textsubscript{2}} respectivamente. Dichas mediciones,
    realizadas con la ventana \textbf{Hanning}, se pueden ver en la Figura~\ref{fig:MedicionIMD}.

    \begin{figure}[H]
      \centering
      \begin{subfigure}[H]{0.48\textwidth}
        \frame{\includegraphics[width=\textwidth]{Imagenes/ActividadPractica/5ProductosDeIMPde3erOrden/Exp5_IDM3erOrdenConSeñal49kHz.jpeg}}
        \caption{Para $f_1=49~kHz$.}
      \end{subfigure}
      \hfill 
      \begin{subfigure}[H]{0.48\textwidth}
        \frame{\includegraphics[width=\textwidth]{Imagenes/ActividadPractica/5ProductosDeIMPde3erOrden/Exp5_IDM3erOrdenConSeñal51kHz.jpeg}}
        \caption{Para $f_2=51~kHz$.}
      \end{subfigure}

      \caption{Medición de diferencia de amplitudes.}
      \label{fig:MedicionIMD}
    \end{figure}

    Los valores obtenidos de esta experiencia se encuentran tabulados en la Tabla~\ref{tab:DatosDeMedicionDeIMD}.

    \begin{table}[H]
      \centering
    \begin{tabular}{ccccc} \hline \hline
      $\mathbf{f_{1}}$    &   $\mathbf{f_{2}}$  &  $\mathbf{2f_{1}-f_{2}}$  & $\mathbf{2f_{2}-f_1}$  & \textbf{Rechazo IMD 3º}\\ \hline
      $49~kHz$   &   $51~kHz$   &    $47~kHz$   &   $53~kHz$  & $28~dB$ \\ \hline \hline
      \end{tabular}
      \caption{Valores obtenidos para la medición del rechazo de IMD de 3º.}
      \label{tab:DatosDeMedicionDeIMD}
    \end{table}


    \pagebreak


