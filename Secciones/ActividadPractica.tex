  \pagebreak
  \section{Actividad Práctica}
    Se propone como actividad realizar el análisis de distintas señales en la frecuencia con el uso del
    módulo matemático de un osciloscopio digital, mediante la FFT. Con este objetivo, los instrumentos
    e insumos necesarios son:

    \begin{itemize}
      \item Osciloscopio digital Tektronix TDS 1001
      \item 2 generadores de señal Goldstar FG-8002
      \item Multímetro RMS MARCA Y MODELOOOOOOO
      \item Circuito modulador de amplitud con diodo y circuito sintonizado
      \item Potenciómetro de $1~k\Omega$
    \end{itemize}

      \subsection{Análisis de una forma de onda cuadrada}

      \subsection{Análisis de un tren de pulsos}

    Se setea la señal de entrada.

        \begin{figure}[H]
        \centering
        \begin{subfigure}[H]{0.48\textwidth}
          \frame{\includegraphics[width=\textwidth]{Imagenes/ActividadPractica/2AnalisisDeUnTrenDePulsos/Exp2_PeriodoDeLaSeñalDeEntrada.jpeg}}
          \caption{Señal pulsante de entrada, período de $1~ms$.}
        \end{subfigure}
        \hfill
        \begin{subfigure}[H]{0.48\textwidth}
          \frame{\includegraphics[width=\textwidth]{Imagenes/ActividadPractica/2AnalisisDeUnTrenDePulsos/Exp2_AnchoDelPulsoDeEntrada.jpeg}}
          \caption{Ancho del pulso.}
        \end{subfigure}

        \caption{Señal pulsante de entrada.}
        \label{fig:Exp2SeñalPulso}
      \end{figure}

    Se observa en el espectro.

      \begin{figure}[H]
        \centering
          \frame{\includegraphics[width=0.48\textwidth]{Imagenes/ActividadPractica/2AnalisisDeUnTrenDePulsos/Exp2_EspectroDeLaSeñalPulsante.jpeg}}
          \caption{Análisis en frecuencia de la señal de entrada.}
          \label{fig:Exp2SeñalPulsanteEspectro}
      \end{figure}

      Luego se conmuta en los tres tipos de ventana.

      \begin{figure}[H]
        \centering
        \begin{subfigure}[H]{0.48\textwidth}
          \frame{\includegraphics[width=\textwidth]{Imagenes/ActividadPractica/2AnalisisDeUnTrenDePulsos/Exp2_EspectroEnVentanaHannin.jpeg}}
          \caption{Análisis en frecuencia, ventana Hanning.}
        \end{subfigure}
        \hfill
        \begin{subfigure}[H]{0.48\textwidth}
          \frame{\includegraphics[width=\textwidth]{Imagenes/ActividadPractica/2AnalisisDeUnTrenDePulsos/Exp2_EspectroEnVentanaRectangular.png}}
          \caption{Análisis en frecuencia, ventana Rectangular.}
        \end{subfigure}
        \begin{subfigure}[H]{0.48\textwidth}
          \frame{\includegraphics[width=\textwidth]{Imagenes/ActividadPractica/2AnalisisDeUnTrenDePulsos/Exp2_EspectroEnVentanaFlattop.jpeg}}
          \caption{Análisis en frecuencia, ventana Flattop.}
        \end{subfigure}

        \caption{Señal pulsante de entrada.}
        \label{fig:Exp2SeñalPulsanteVentanasEspectro}
      \end{figure}

      Se observan los 5 primeros armónicos en ventana Hanning.

       \begin{figure}[H]
        \centering
        \begin{subfigure}[H]{0.48\textwidth}
          \frame{\includegraphics[width=\textwidth]{Imagenes/ActividadPractica/2AnalisisDeUnTrenDePulsos/Exp2_FrecArmonico1.jpeg}}
          \caption{Frecuencia de la fundamental en ventana Hanning, $f_{1}=950~Hz$.}
        \end{subfigure}
        \hfill
        \begin{subfigure}[H]{0.48\textwidth}
          \frame{\includegraphics[width=\textwidth]{Imagenes/ActividadPractica/2AnalisisDeUnTrenDePulsos/Exp2_FrecArmonico2.jpeg}}
          \caption{Frecuencia de la segunda armónica en ventana Hanning, $f_{2}=1950~Hz$.}
        \end{subfigure}
        \begin{subfigure}[H]{0.48\textwidth}
          \frame{\includegraphics[width=\textwidth]{Imagenes/ActividadPractica/2AnalisisDeUnTrenDePulsos/Exp2_FrecArmonico3.jpeg}}
          \caption{Frecuencia de la tercera armónica en ventana Hanning, $f_{3}=2950~Hz$.}
        \end{subfigure}
        \begin{subfigure}[H]{0.48\textwidth}
          \frame{\includegraphics[width=\textwidth]{Imagenes/ActividadPractica/2AnalisisDeUnTrenDePulsos/Exp2_FrecArmonico4.jpeg}}
          \caption{Frecuencia de la cuarta armónica en ventana Hanning, $f_{4}=3950~Hz$.}
        \end{subfigure}
        \begin{subfigure}[H]{0.48\textwidth}
          \frame{\includegraphics[width=\textwidth]{Imagenes/ActividadPractica/2AnalisisDeUnTrenDePulsos/Exp2_FrecArmonico5.jpeg}}
          \caption{Frecuencia de la quinta armónica en ventana Hanning, $f_{5}=4900~Hz$.}
        \end{subfigure}

        \caption{Medición de frecuencia de armónicos de la señal pulsante en ventana Hanning.}
        \label{fig:Exp2SeñalPulsanteArmonicosEspectro}
      \end{figure}     

      Luego se pasa a Flattop y se miden las frecuencias de los valles.

       \begin{figure}[H]
        \centering
        \begin{subfigure}[H]{0.48\textwidth}
          \frame{\includegraphics[width=\textwidth]{Imagenes/ActividadPractica/2AnalisisDeUnTrenDePulsos/Exp2_FrecValle1Flattop.jpeg}}
          \caption{Frecuencia del primer valle en ventana Flattop, $f_{a}=400~Hz$.}
        \end{subfigure}
        \hfill
        \begin{subfigure}[H]{0.48\textwidth}
          \frame{\includegraphics[width=\textwidth]{Imagenes/ActividadPractica/2AnalisisDeUnTrenDePulsos/Exp2_FrecValle2Flattop.jpeg}}
          \caption{Frecuencia del segundo valle en ventana Rectangular, $f_{b}=1450~Hz$.}
        \end{subfigure}
        \hfill
        \begin{subfigure}[H]{0.48\textwidth}
          \frame{\includegraphics[width=\textwidth]{Imagenes/ActividadPractica/2AnalisisDeUnTrenDePulsos/Exp2_FrecValle3Flattop.jpeg}}
          \caption{Frecuencia del tercer valle en ventana Flattop, $f_{c}=2450~Hz$.}
        \end{subfigure}

        \caption{Medición de frecuencia de valles de la señal pulsante en ventana Flattop.}
        \label{fig:Exp2SeñalPulsanteVallesEspectro}
      \end{figure}     

      Se hace una tabla para la ventana Hanning. Se coloca el cursor 1 en 0Hz y el otro marca el 
      armónico.

      \begin{table}[H]
      \centering
        \begin{tabular}{cccccc} \hline \hline
          \textbf{Cursor 2}               &  $\mathbf{1erArm.}$       & $\mathbf{2daArm.}$        & $\mathbf{3raArm.}$  &   $\mathbf{4taArm.}$ &   $\mathbf{5taArm.}$ \\ \hline
          $\mathbf{\Delta_{fn}~Hz}$       &   $950$                        &    $950$                    &   $950$                & $950$                   & $950$                \\
         \end{tabular}
          \caption{Valores de frecuencia medidos en ventan Hanning.}
          \label{tab:Exp1MedicionesHanning}
      \end{table}

      Se calcula el promedio de las frecuencias.

      \begin{align*}
        \Delta_{fn_{prom}}=\dfrac{\sum{\Delta_{fn}}}{n} \hspace{20pt} \therefore \hspace{20pt} \boxed{\Delta_{fn_{prom}}=950~[Hz]}
      \end{align*}

      Y el período de la onda de pulsos es.

      \begin{align*}
        Periodo~\left( T \right)=\dfrac{1}{\Delta_{fn_{prom}}} \hspace{20pt} \therefore \hspace{20pt} \boxed{Periodo~\left( T \right)=1,053~[ms]}
      \end{align*}

      Se cambia a flattop y se miden los valles.

      \begin{table}[H]
      \centering
        \begin{tabular}{cccc} \hline \hline
          \textbf{Cursor 2}               &  $\mathbf{1erArm.}$       & $\mathbf{2daArm.}$        & $\mathbf{3raArm.}$   \\ \hline
          $\mathbf{\Delta_{fmin}~Hz}$       &   $400$                        &    $1050$                    &   $1000$                     \\
         \end{tabular}
          \caption{Valores de frecuencia medidos en ventan Flattop.}
          \label{tab:Exp1MedicionesFlattop}
      \end{table}  

      También se calcula promedio y período.

      \begin{align*}
        \Delta_{fn_{prom}}=\dfrac{\sum{\Delta_{fn}}}{n} \hspace{20pt} \therefore \hspace{20pt} \boxed{\Delta_{fn_{prom}}=816,67~[Hz]}
      \end{align*}        

      \begin{align*}
        Periodo~\left( T \right)=\dfrac{1}{\Delta_{fn_{prom}}} \hspace{20pt} \therefore \hspace{20pt} \boxed{Periodo~\left( T \right)=1,224~[ms]}
      \end{align*}
      \pagebreak
  \subsection{Obervación de frecuencias producto del aliasing}
    El \textit{aliasing} es un fenómeno que produce componentes de frecuencia falsas, es decir,
    que nos son propias de la señal que se desea medir. Esto, en los osciloscopios digitales,
    ocurre cuando la velocidad de muestreo que se utiliza es insuficiente, es decir, muy baja,
    para la señal que se desea medir.
    
    Por el teorema del muestreo se sabe que la velocidad de muestreo debe ser, como mínimo, el
    doble de la frecuencia máxima de la señal a ser medida.

    \subsubsection{Práctica}
      Con el uso de uno de los generadores se setea una señal \textbf{senoidal} de frecuencia
      $\mathbf{f = 10~kHz}$, y una amplitud acorde a la mitad del valor posible, ya que no es
      algo crítico. Luego, con el menú matemático se eligen las opciones \textbf{FFT}, \textbf{CH1},
      \textbf{Hanning} y \textbf{Zoom X1}. Además, la velocidad de muestreo se setea en 
      $\mathbf{f_s = 25~kSa/s}$. En la Figura~\ref{fig:Exp3Señal10k} se puede observar que la señal 
      es correctamente muestreada, ya que se logra ver su componente de frecuencia.
      
      \begin{figure}[H]
        \centering
        \frame{\includegraphics[width=0.48\textwidth]{Imagenes/ActividadPractica/3ObservacionDeAlias/Exp3_Señal10K.jpeg}}
        \caption{Espectro de señal senoidal de 10~kHz.}
        \label{fig:Exp3Señal10k}
      \end{figure}
      
      Ahora, se procede a cambiar la frecuencia de la señal que se desea medir, y se logra ver que
      a partir de la frecuencia $\mathbf{f = 12,5~kHz}$ empiezan a haber frecuencias falsas. Este
      efecto es esperable, ya que la velocidad de muestreo elegida cumple con el teorema de muestreo
      solo para señales senoidales cuya frecuencia, como máximo, sea la mencionada en este párrafo.

      \begin{figure}[H]
        \centering
        \begin{subfigure}[H]{0.48\textwidth}
          \frame{\includegraphics[width=\textwidth]{Imagenes/ActividadPractica/3ObservacionDeAlias/Exp3_SeñalConAliasing.jpeg}}
          \caption{$f=12,5~kHz$ y $fs=25~kSa/s$.}
          \label{fig:Exp3AliasSeñal12_5kHz}
        \end{subfigure}
        \hfill 
        \begin{subfigure}[H]{0.48\textwidth}
          \frame{\includegraphics[width=\textwidth]{Imagenes/ActividadPractica/3ObservacionDeAlias/Exp3_SeñalConAliasing.jpeg}}
          \caption{$f=15~kHz$ y $fs=25~kSa/s$.}
          \label{fig:Exp3AliasSeñal15kHz}
        \end{subfigure}

        \caption{Espectro de señales con aliasing.}
        \label{fig:Exp3SeñalConAlias}
      \end{figure}

      \subsubsection*{Aliasing con señal cuadrada}
      Se procede a hacer la misma experiencia anterior, pero con una señal cuadrada de $\mathbf{f=10kHz}$.
      Su forma en el tiempo y su espectro se puede ver en la Figura~\ref{fig:Exp3SeñalCuad10k}.

      \begin{figure}[H]
        \centering
        \begin{subfigure}[H]{0.48\textwidth}
          \frame{\includegraphics[width=\textwidth]{Imagenes/logo-utn.png}}
          \caption{$f=10~kHz$ .}
          \label{fig:Exp3SeñalCuadEnTiempo}
        \end{subfigure}
        \hfill 
        \begin{subfigure}[H]{0.48\textwidth}
          \frame{\includegraphics[width=\textwidth]{Imagenes/ActividadPractica/3ObservacionDeAlias/Exp3_SeñalCuadradaSINAlias.jpeg}}
          \caption{$f=15~kHz$ y $fs=25~kSa/s$.}
          \label{fig:Exp3EspectroSeñalCuad}
        \end{subfigure}

        \caption{Espectro de señales con aliasing.}
        \label{fig:Exp3SeñalCuad10k}
      \end{figure}

      Luego, se procede a aumentar la frecuencia de la señal del generador, y también se logra ver los efectos
      del aliasing. En la Figura~\ref{fig:Exp3SeñalCuadConAlias} se muestra lo mencionado.

      \begin{figure}[H]
        \centering
        \begin{subfigure}[H]{0.48\textwidth}
          \frame{\includegraphics[width=\textwidth]{Imagenes/ActividadPractica/3ObservacionDeAlias/Exp3_SeñalConAliasing.jpeg}}
          \caption{$f=12,5~kHz$ y $fs=25~kSa/s$.}
          \label{fig:Exp3AliasSeñalCuad1}
        \end{subfigure}
        \hfill 
        \begin{subfigure}[H]{0.48\textwidth}
          \frame{\includegraphics[width=\textwidth]{Imagenes/ActividadPractica/3ObservacionDeAlias/Exp3_SeñalConAliasing.jpeg}}
          \caption{$f=15~kHz$ y $fs=25~kSa/s$.}
          \label{fig:Exp3AliasSeñalCuad2}
        \end{subfigure}

        \caption{Espectro de señales con aliasing.}
        \label{fig:Exp3SeñalCuadConAlias}
      \end{figure}


      \subsection{Análisis de una señal modulada en amplitud}

      \subsection{Observación de los productos de IMD de tercer orden}

      \subsection{Análisis de una señal modulada en frecuencia}

      \subsection{Análisis de la distorsión armónica producida por un amplificador}
    Se propone analizar un amplificador transistorizado de 4 etapas. Se ha tratado 
    en las experiencias anteriores la \textbf{alinealidad} de éste tipo de dispositivos, 
    para el presente experimento, nuevamente se debe tener en cuenta. 

    Un amplificador tiene comportamiento lineal bajo determinadas condiciones, entre ellas, 
    para pequeña señal y trabajando a lazo cerrado. Sucede que al trabajar en lazo abierto y 
    a máxima excursión, se ingresa en zonas no lineales de la función de transferencia, lo que 
    modifica el comportamiento del amplificador, provocando lo que se denomina 
    \textbf{Distorsión Armónica}.

    En la experiencia se determina el porcentaje de contenido armónico, tanto para lazo 
    abierto como para lazo cerrado, utilizando la 
    herramienta de análisis en frecuencia del osciloscopio. El circuito a implementar 
    se enseña en la Figura~\ref{fig:Exp7EsquemaCircuito}.

      \begin{figure}[H]
        \centering
          \frame{\includegraphics[width=0.8\textwidth]{Imagenes/ActividadPractica/7AnalisisDeDistorisonArmonicaEnAmpli/EsquemaConexionAmplificador.png}}
          \caption{Esquema del circuito a implementar.}
          \label{fig:Exp7EsquemaCircuito}
      \end{figure}

    En la Figura~\ref{fig:Exp7Circuito} se muestran los dispositivos y el 
    instrumental a usar.

      \begin{figure}[H]
        \centering
        \begin{subfigure}[H]{0.48\textwidth}
          \frame{\includegraphics[width=0.8\textwidth]{Imagenes/ActividadPractica/7AnalisisDeDistorisonArmonicaEnAmpli/Amplificador.jpeg}}
          \caption{Amplificador a utilizar.}
          \label{fig:Exp7Amplificador}
        \end{subfigure}
        \hfill 
        \begin{subfigure}[H]{0.48\textwidth}
          \frame{\includegraphics[width=0.8\textwidth]{Imagenes/ActividadPractica/7AnalisisDeDistorisonArmonicaEnAmpli/Equipo.jpeg}}
          \caption{Instrumental a utilizar.}
          \label{fig:Exp7Instrumental}
        \end{subfigure} 

          \caption{Amplificador a utilizar.}
          \label{fig:Exp7Circuito}
      \end{figure} 

    Inicialmente se configura el generador para \textbf{MES} (máxima excursión simétrica), y 
    una frecuencia de $\mathbf{1~kHz}$. Para ello se procede a variar la amplitud del generador hasta 
    el punto donde hay un recorte visible en la señal. La Figura~\ref{fig:Exp7MESLazoAbierto} 
    muestra dicho punto de máxima excursión.

      \begin{figure}[H]
        \centering
          \frame{\includegraphics[width=0.48\textwidth]{Imagenes/ActividadPractica/7AnalisisDeDistorisonArmonicaEnAmpli/Exp7_MESlazoAbierto.png}}
          \caption{Máxima excursión simétrica a lazo abierto.}
          \label{fig:Exp7MESLazoAbierto}
      \end{figure}

      Luego, se procede a hacer un análisis de frecuencia, para ello se configura el osciloscopio nuevamente 
      en modo \textbf{FFT}, \textbf{Flattop}, \textbf{Zoom X10}, y $\mathbf{100~kS/s}$. En éste punto, se procede 
      a realizar la medición de la frecuencia de la fundamental y sus dos primeros armónicos, lo cual se observa en la 
      Figura~\ref{fig:Exp7FrecFundYArmonicas}.

      \begin{figure}[H]
        \centering
        \begin{subfigure}[H]{0.48\textwidth}
          \frame{\includegraphics[width=\textwidth]{Imagenes/ActividadPractica/7AnalisisDeDistorisonArmonicaEnAmpli/Exp7_EspectroLazoAbiertoConCursorEn1k.png}}
          \caption{Frecuencia de la fundamental $f_{fund}=1~kHz$.}
          \label{fig:Exp7FrecFundamental}
        \end{subfigure}
        \hfill 
        \begin{subfigure}[H]{0.48\textwidth}
          \frame{\includegraphics[width=\textwidth]{Imagenes/ActividadPractica/7AnalisisDeDistorisonArmonicaEnAmpli/Exp7_FrecuenciasDeLaSegundaYTercerArmonica.png}}
          \caption{Frecuencias de la segunda y tercer armónica $f_{2}=2~kHz$ y $f_{3}=3~kHz$.}
          \label{fig:Exp7Frec2da3raArmo}
        \end{subfigure}     
        \caption{Análisis espectral del amplificador con señal de $1~kHz$ a lazo abierto.}
        \label{fig:Exp7FrecFundYArmonicas}
      \end{figure}

      En la Figura~\ref{fig:Exp7AmplFundYArmonicasLA}, se miden las amplitudes de la fundamental y sus 
      dos primeros armónicos.

    \begin{figure}[H]
        \centering
        \begin{subfigure}[H]{0.48\textwidth}
          \frame{\includegraphics[width=\textwidth]{Imagenes/ActividadPractica/7AnalisisDeDistorisonArmonicaEnAmpli/Exp7_AmplitudDeLaFundamental.png}}
          \caption{Fundamental $V_{fund}=5,03~dBv$.}
          \label{fig:Exp7AmpFundamentalLA}
        \end{subfigure}
        \hfill 
        \begin{subfigure}[H]{0.48\textwidth}
          \frame{\includegraphics[width=\textwidth]{Imagenes/ActividadPractica/7AnalisisDeDistorisonArmonicaEnAmpli/Exp7_AmplitudDelSegundoArmónico.png}}
          \caption{2da armónica $V_{2da}=-37,4~dBv$.}
          \label{fig:Exp7AmpSegundaLA}
        \end{subfigure}     
        \begin{subfigure}[H]{0.48\textwidth}
          \frame{\includegraphics[width=\textwidth]{Imagenes/ActividadPractica/7AnalisisDeDistorisonArmonicaEnAmpli/Exp7_AmplitudDelTercerArmónicoALazoAbierto.png}}
          \caption{3ra armónica $V_{2da}=-47~dBv$.}
          \label{fig:Exp7AmpTercerLA}
        \end{subfigure}   
        \caption{Amplitud de la fundamental y sus dos primeras armónicas a lazo abierto.}
        \label{fig:Exp7AmplFundYArmonicasLA}
      \end{figure}

      Ahora, se procede a hacer el mismo análisis a lazo cerrado. Para empezar, se restituye el nivel de la 
      fundamental, y se ubican las frecuencias 
      de la misma y sus dos primeros armónicos, ésto se enseña en la Figura~\ref{fig:Exp7FrecLazoCerrado}.

      \begin{figure}[H]
        \centering
          \frame{\includegraphics[width=0.48\textwidth]{Imagenes/ActividadPractica/7AnalisisDeDistorisonArmonicaEnAmpli/Exp7_LazoCerradoFrecuenciasDeSegundoYTercerArmonico.png}}
          \caption{Frecuencia de armónicos a lazo cerrado.}
          \label{fig:Exp7FrecLazoCerrado}
      \end{figure}

      Por consiguiente, se miden las amplitudes de los ya mencionados, como muestra la 
      Figura~\ref{fig:Exp7AmplFundYArmonicasLC}. Para ello se coloca un cursor en el punto 
      a medir, y éste valor es correspondiente a la amplitud en dB referenciada a $1~[V_{rms}]$.

      \begin{figure}[H]
        \centering
        \begin{subfigure}[H]{0.48\textwidth}
          \frame{\includegraphics[width=\textwidth]{Imagenes/ActividadPractica/7AnalisisDeDistorisonArmonicaEnAmpli/Exp7_LazoCerradoAmplitudDelPrimerArmonico.png}}
          \caption{Fundamental $V_{fund}=66~dBv$.}
          \label{fig:Exp7AmpFundamentalLC}
        \end{subfigure}
        \hfill 
        \begin{subfigure}[H]{0.48\textwidth}
          \frame{\includegraphics[width=\textwidth]{Imagenes/ActividadPractica/7AnalisisDeDistorisonArmonicaEnAmpli/Exp7_LazoCerradoAmplitudDelSegundoArmonico.png}}
          \caption{2da armónica $V_{2da}=18,4~dBv$.}
          \label{fig:Exp7AmpSegundaLC}
        \end{subfigure}     
        \begin{subfigure}[H]{0.48\textwidth}
          \frame{\includegraphics[width=\textwidth]{Imagenes/ActividadPractica/7AnalisisDeDistorisonArmonicaEnAmpli/Exp7_LazoCerradoAmplitudDelTercerArmónico.png}}
          \caption{3ra armónica $V_{2da}=16,4~dBv$.}
          \label{fig:Exp7AmpTercerLC}
        \end{subfigure}   
        \caption{Amplitud de la fundamental y sus dos primeras armónicas a lazo cerrado.}
        \label{fig:Exp7AmplFundYArmonicasLC}
      \end{figure}    

      Finalmente se confecciona una tabla de valores con los resultados obtenidos para lazo abierto y 
      lazo cerrado, y se calcula la 
      \textbf{Distorsión armónica} (DA) en porcentaje, que se define como 
        \begin{equation}
          DA_{\%}= \dfrac{V_{arm}}{V_{1}} \cdot 100~, \vspace{10pt} con~V_{arm}=\sqrt[]{V_{2da}^2+V_{3ra}^2}~.
          \label{eqn:DistorsionArmonica}
        \end{equation}
      
      Lo anteriormente dicho se muestra en la 
      Tabla~\ref{tab:Exp7DistArmLA} y la Tabla~\ref{tab:Exp7DistArmLC}.

      \begin{table}[H]
      \centering
        \begin{tabular}{ccccc} \hline \hline
          \textbf{Magnitud}            &   $\mathbf{V_{salida(PAP)}}$  &  $\mathbf{1erArmonica}$  & $\mathbf{2daArmonica}$  & $\mathbf{3raArmonica}$\\ \hline \hline
          \textbf{Frecuencia~[kHz]}    &   $1$                         &    $1$                   &   $2$                   & $3$ \\ \hline
          \textbf{dBv}                 &   $-$                         &    $5,03$                &   $-37,4$                & $-47$ \\ \hline
          \textbf{Tensión~[V]}         &   $4,88$                      &    $1,784$             &   $0,0135$              & $0,00447$\\ \hline \hline
          \end{tabular}
          \caption{Valores obtenidos a Lazo Abierto.}
          \label{tab:Exp7DistArmLA}
      \end{table}

     \begin{table}[H]
      \centering
        \begin{tabular}{ccccc} \hline \hline
          \textbf{Magnitud}            &   $\mathbf{V_{salida(PAP)}}$  &  $\mathbf{1erArmonica}$  & $\mathbf{2daArmonica}$  & $\mathbf{3raArmonica}$\\ \hline \hline
          \textbf{Frecuencia~[kHz]}    &   $1$                         &    $1$                   &   $2$                   & $3$ \\ \hline
          \textbf{dBv}                 &   $-$                         &    $5,03$                &   $-42,6$                & $-44,6$ \\\hline
          \textbf{Tensión~[V]}         &   $4,88$                      &    $1,784$             &   $0,00741$              & $0,00588$\\ \hline \hline
          \end{tabular}
          \caption{Valores obtenidos a Lazo Cerrado.}
          \label{tab:Exp7DistArmLC}
      \end{table}

      Finalmente, se determina la distorsión armónica total a lazo abierto y lazo 
      cerrado. A lazo abierto se tiene que 
      \begin{align*}
        V_{arm_{Lazo Abierto}}=\sqrt[]{(13,5~mV)^2+(4,47~mV)^2} \hspace{20pt} \Longrightarrow \hspace{20pt} V_{arm}=14,22~[mV]~,
      \end{align*}
      y haciendo uso de la ecuación~\ref{eqn:DistorsionArmonica} dicho valor será
      \begin{align*}
        Distorsion~Armonica_{arm_{Lazo Abierto}}= \dfrac{14,22~mV}{1,784~V} \cdot 100 \hspace{20pt} \therefore \hspace{20pt} \boxed{Distorsion_{LA}=0,8\%}~.
      \end{align*}

      Luego se repiten los cálculos para lazo cerrado, obteniéndose 
      \begin{align*}
        V_{arm_{Lazo Cerrado}}=\sqrt[]{(7,41~mV)^2+(5,88~mV)^2} \hspace{20pt} \Longrightarrow \hspace{20pt} V_{arm}=9,46~[mV]~,
      \end{align*}
      y con éste valor se determina la distorsión armónica total, que será 
      \begin{align*}
        Distorsion~Armonica_{arm_{Lazo Cerrado}}= \dfrac{9,46~mV}{1,784~V} \cdot 100 \hspace{20pt} \therefore \hspace{20pt} \boxed{Distorsion_{LC}=0,53\%}~.
      \end{align*}

