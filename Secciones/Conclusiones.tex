  \pagebreak
  \section{Conclusiones}

    Es posible destacar una ventaja importante del análisis en frecuencia respecto al tiempo de 
    determinadas señales, la cual se relaciona con la visualización del contenido armónico de las 
    mismas, que no siempre está a la vista, principalmente cuando se está analizando señales con 
    una cantidad considerable de superposición armónica (lo que dificulta el análisis en el dominio 
    del tiempo), o también para develar el de una señal que aparenta no tener ésto último. En 
    contraposición, una desventaja igual de considerable es la dificultad que presenta el análisis 
    en frecuencia para determinar transitorios de corta duración.

  Respecto a los distintos tipos de ventanas que se encuentran disponibles para la FFT, se puede realizar
  las siguientes afirmaciones:

  \begin{itemize}
    \item \textbf{Hanning:} es útil para las formas de onda periódicas. Permite medir mejor la frecuencia, pero peor
      la amplitud que la ventana Flattop.
    \item \textbf{Flattop:} es útil para formas de onda periódica. Permite medir mejor la amplitud, pero peor
      la frecuencia que la ventana Hanning.
    \item \textbf{Rectangular:} es útil para pulsos o señales transitorias. Es específica para formas de onda
      que no presentan discontinuidades.
  \end{itemize}

  En el experimento 3 se forzó la aparición del aliasing. Esto se dio debido a que la velocidad de muestreo ($25~kSa/s$)
  no cumplía con el teorema del Muestreo a partir de determinada frecuencia de la señal a muestrear (a partir de $12~kHz$).
  Esto significa, por un lado, que la señal no puede ser muestreada de forma correcta, y además, se generan componentes
  de frecuencia falsas que no son propias de la señal original.

  Para concluir, se nombran algunas características sobre los distintos modos de adquisición que posee el osciloscopio
  digital:

  \begin{itemize}
    \item \textbf{Normal}: el osciloscopio muestrea la señal en intervalos de tiempo equidistantes para construir la señal. Este
      modo representa de forma acertada señales analógicas sin transitorios rápidos.
    \item \textbf{Detección de picos}: el osciloscopio detecta los valores máximo y mínimo de la señal de entrada en un intervalo
      de muestreo, y los utiliza para construir la señal. Con este modo se pueden detectar pulsos transitorios, que con el
      modo Normal podrían perderse.
    \item \textbf{Promedio}: el osciloscopio adquiere determinada cantidad de veces la señal, promedia esos datos y construye la
      señal con el resultado. Es útil para reducir el ruido.
  \end{itemize}

